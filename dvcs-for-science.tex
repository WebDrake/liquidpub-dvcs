% dvcs-for-science.tex
%
% Copyright (c) 2009 Joseph Rushton Wakeling
%
% This file is part of 'Distributed version control for credit and collaboration
% in science', a Liquid Document that may be redistributed under any or all of
% the following three licenses:
%
%  -- the GNU Affero General Public License as published by the
%     Free Software Foundation, either version 3 of the license
%     or (at your option) any later version.
%
%  -- the Creative Commons Attribution-Share Alike License as
%     published by Creative Commons, either version 3.0 of the
%     license or (at your option) any later version.
%
%  -- the Creative Commons Attribution License as published by
%     Creative Commons, either version 3.0 of the license or
%     (at your option) any later version.
%
% You should have received copies of the GNU Affero General Public License, the
% Creative Commons Attribution-Share Alike License and the Creative Commons
% Attribution License along with this document.  If not, see
% <http://www.gnu.org/licenses/> and <http://creativecommons.org/licenses/>.
%
% For more information on the Liquid Publication Project, visit
% <http://project.liquidpub.org/>.


\documentclass[rmp,twocolumn]{revtex4}
\usepackage[dvips]{graphicx}
\usepackage{times}

\begin{document}
\title{Distributed version control for credit and collaboration in science}
\begin{abstract}
We investigate the question of how scientific research can become more openly
collaborative, in the sharing spirit of free and open source software.  In
particular we examine how distributed version control tools can be used to
manage an open community of researchers who, while sharing common research
goals, may differ strongly in their interpretation of results or their ideas
about the most productive lines of investigation.  At the same time, we examine
the question of how credit can be meaningfully assigned to contributors---credit
for concrete contributions of results or writing, for ideas, and for
organisational work such as project maintenance.  This article is itself being
developed in a collaborative DVCS system with open contributions solicited.  The
experience of contributors in working on the article will feed its own results
and conclusions.
\end{abstract}

\maketitle

% introduction.tex
%
% Copyright (c) 2009 Joseph Rushton Wakeling
%
% This file is part of 'Distributed version control for credit and collaboration
% in science', a Liquid Document that may be redistributed under any or all of
% the following three licenses:
%
%  -- the GNU Affero General Public License as published by the
%     Free Software Foundation, either version 3 of the license
%     or (at your option) any later version.
%
%  -- the Creative Commons Attribution-Share Alike License as
%     published by Creative Commons, either version 3.0 of the
%     license or (at your option) any later version.
%
%  -- the Creative Commons Attribution License as published by
%     Creative Commons, either version 3.0 of the license or
%     (at your option) any later version.
%
% You should have received copies of the GNU Affero General Public License, the
% Creative Commons Attribution-Share Alike License and the Creative Commons
% Attribution License along with this document.  If not, see
% <http://www.gnu.org/licenses/> and <http://creativecommons.org/licenses/>.
%
% For more information on the Liquid Publication Project, visit
% <http://project.liquidpub.org/>.


\section{Introduction}

% overview-dvcs.tex
%
% Copyright (c) 2009 Joseph Rushton Wakeling
%
% This file is part of 'Distributed version control for credit and collaboration
% in science', a Liquid Document that may be redistributed under any or all of
% the following three licenses:
%
%  -- the GNU Affero General Public License as published by the
%     Free Software Foundation, either version 3 of the license
%     or (at your option) any later version.
%
%  -- the Creative Commons Attribution-Share Alike License as
%     published by Creative Commons, either version 3.0 of the
%     license or (at your option) any later version.
%
%  -- the Creative Commons Attribution License as published by
%     Creative Commons, either version 3.0 of the license or
%     (at your option) any later version.
%
% You should have received copies of the GNU Affero General Public License, the
% Creative Commons Attribution-Share Alike License and the Creative Commons
% Attribution License along with this document.  If not, see
% <http://www.gnu.org/licenses/> and <http://creativecommons.org/licenses/>.
%
% For more information on the Liquid Publication Project, visit
% <http://project.liquidpub.org/>.


\section{An overview of distributed revision control}


\end{document}
